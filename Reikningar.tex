\section{Reikningar}\label{ch::reikningar}

Við reikninga var miðað við eina plötu, \te þegar aðskilnaður flata hefur átt sér stað. 

Byrjað var á ytri greiningu samsetningarinnar. 
Þar sem ytra álag á samsetninguna er samhverft um miðju fæst að báðir undirstöðukraftarnir eru $R_{A} = R_{B} = F/2$, \sbr mynd \ref{fig::forsendur}. 
Við útreikninga var miðað við að mestu spennur sem boltinn eða efnið þyldi væru flotmörk, \te $n_y = 1$. 
Mesta beygjuspenna sem hver plata verður fyrir er í miðju samsetningarinnar, út frá því má finna mesta kraftinn sem hver plata þolir án þess að fara í flot.

\begin{equation}
  \sigma = {M c \over I} = {F l h \over 4I} 
  \Rightarrow F = {4 \sigma I \over l h}
  \label{eq::1}
\end{equation}

Í jöfnu \ref{eq::1} eru allar stærðir fastar nema flatartregðuvægið, I. 
Með því að lækka flatartregðuvægið mun mesti kraftur sem platan þolir fyrir flot minnka og því er ákveðið að bora ekki í miðja plötuna. 
Þessi kraftur, $F_{y,p}$, er því notaður sem lágmarksviðmið þegar stærð, staðsetning og fjöldi bolta er valin. 
Forsendur fyrir því að einhver tiltekin hönnunin virki er því að samsetningin má ekki fara í flot við minni kraft en $F_{y,p}$.

Flatartregðuvægið fyrir ferhyrnt þversnið er $I = {1 \over 12}bh^3$, því fæst að kraftur sem veldur því að samsetningin fari í flot í miðju:

\[
F_{y,p} 
= 
{
  4  \sigma_y I 
  \over 
  l h
}
= 
{
  \sigma_y b h^2 
  \over 
  3 l
} 
= 
{
  210 MPa \cdot 5 mm \cdot \left( 50 mm \right)^2 
  \over 
  3 \cdot 200mm
} 
= 4375N
\]

\subsection{Mismunandi tilvik semsetninga}

Plötur get vissulega farið í flot, en auk þess þarf að skoða þrjú önnur tilvik:

\begin{enumerate}
\item Beygjuspenna á yfirborði í þeirri línu, þvert á samsetninguna, sem götin eru boruð
\item Skerspenna sem boltar verða fyrir
\item Leguspenna sem verkar á efnið í götunum
\end{enumerate}

Til að minnka beygjuspennu á yfirborði er ákveðið að hafa götin sem næst lóðréttum brúnum samsetningarinnar.
Þannig er vægisarmurinn frá undirstöðunum styttur, sem minnkar beygjuspennuna \sbr jöfnu \ref{eq::1}. 
Við þetta hækkar skerspennan í boltum og leguspennan í götum efnisins. 
Til að koma til móts við þessa spennuhækkun er ákveðið að hafa götin sem fjærst láréttum brúnum samsetningarinnar, við það munu þær spennur lækka, \sbr jöfnu \ref{eq::2}\cite{shigleys}.

\begin{equation}
  F_n^{''} = {
    M r_n 
    \over 
    \displaystyle\sum_{i=1}^{k} r_i^2
  }
  \label{eq::2}
\end{equation}

$F_n^{''}$ er skerkraftur sem verkar á bolta n sem er í fjarlægð $r_n$ frá flatarmiðju boltanna og k er fjöldi bolta í samsetningunni. 
Með þessu er hægt að finna leguspennu sem verkar á efnið og skerspennu sem verkar á boltana. 
Öll þrjú ofangreind tilvik voru tekin fyrir og skoðaðar þrjár samsetningar með tilliti til þeirra.
Í hverju tilfelli fyrir sig var krafturinn sem olli því að fyrrnefndar spennur færu í flot reiknaður.

\subsubsection{Fyrsta tilvik}

Í upphafi var einföld hönnun valin, \sbr mynd \ref{fig::2x6}. Vegna samhverfu um miðju er nóg að reikna aðeins fyrir annan boltann.

\begin{itemize}
\item Beygjuspenna á yfirborði við göt
  
  Jafna \ref{eq::1} er notuð við þetta en nú er flatartregðuvægið minna vegna holunnar. 
  Flatartregðuvægið er því $I = {1 \over 12}bh^3 - {1 \over 12}bd^3$. Því er sá kraftur sem veldur floti
  \[
  F = {4 \sigma I \over l h} 
  = {
    4 \cdot 210 MPa \cdot 
    \left(
      {
        1 \over 12
      } 
      \cdot 5 mm \cdot 
      \left(
        50mm
      \right)^3
      - 
      {
        1 \over 12
      } 
      5 mm \cdot 
      \left(
        6mm
      \right)^3
    \right) 
    \over 
    (160 mm + 12 mm) \cdot 50 mm
  } 
  = \underline{5080N}
  \] 

\item Skerspenna í bolta

  Jafna \ref{eq::2} er notuð til að finna skerkraftinn sem virkar á boltann

  \[
  F_1^{''} 
  = F_2^{''} 
  = {Mr_n \over 2r_n^2} 
  = 
  {
    {F \over 2}l \over 2 r_n
  } 
  \]

  Sá kraftur sem veldur því að skerspennan í boltanum nái flotspennu

  \[
  \sigma 
  = {F_1^{''} \over A_b} 
  \Rightarrow 
  F = 
  {
    4 S_{sy} A_b r_n \over l
  } 
  = {
    4 \cdot 
    {
      660MPa \over \sqrt{3}
    } 
    \cdot 
    {
      \pi \left(6mm\right)^2 \over 4
    } 
    \cdot 
    28 mm \over 200 mm
  } 
  = \underline{6030N}
  \]

\item Leguspenna í efni
  
  Sá kraftur sem virkar á leguna er skerkrafturinn sem verkar á bolta, því verður leguspennan
  
  \[
  \sigma 
  = {F_1^{''} \over bd} 
  \Rightarrow 
  F = {4 \sigma_{y} b d r_n \over l} 
  = 
  {
    4 \cdot 210 MPa \cdot 5 mm \cdot 6 mm \cdot 28 mm 
    \over 
    200mm
  } 
  = \underline{3528N}
  \]
\end{itemize}

Hér að ofan sést að sá kraftur sem veldur floti í legum efnisins er lægri en $F_{y,p}$ og því virkar þessu hönnun ekki miðað við þær forsendur sem við gáfum okkar.

\begin{figure}
  \centering
  \includegraphics[width=0.5\textwidth]{2x6}
  \caption{Fyrsta reiknað tilvik}
  \label{fig::2x6}
\end{figure}

\subsubsection{Annað tilvik}

Þar sem fyrsta tilvik gekk ekki vegna leguspenna var athugað hvort 8 mm gat í stað 6 mm myndi standast forsendur okkar, \sbr mynd \ref{fig::2x8}. Vegna þess að aðeins ein breyting var gerð reiknast annað tilvikið líkt og það fyrsta.

\begin{itemize}
\item Beygjaspenna á yfirborði við göt
  
  \[
  F = {4 \sigma I \over l h} 
  = {
    4 \cdot 210 MPa \cdot 
    \left(
      {1 \over 12} 
      \cdot 5 mm \cdot 
      \left(
        50mm
      \right)^3 
      - 
      {1 \over 12} 5 mm \cdot 
      \left(
        8mm
      \right)^3
    \right) 
    \over 
    (160 mm + 12 mm) \cdot 50 mm
  } 
  = \underline{5066N}
  \] 
  
\item Skerspenna í bolta
  
  Á sama hátt og í fyrsta tilviki fæst sá kraftur sem veldur því að skerspennan í boltanum nái flotspennu
  
  \[
  \sigma = {F_1^{''} \over A_b} 
  \Rightarrow 
  F = {4 S_{sy} A_b r_n \over l} 
  = {4 \cdot {660MPa \over \sqrt{3}} 
    \cdot 
    {
      \pi 
      \left(
        8mm
      \right)^2 
      \over 
      4
    } 
    \cdot 28 mm \over 200 mm
  } 
  = \underline{10726N}
  \]
  
\item Leguspenna í efni
  
  Sá kraftur sem virkar á leguna er skerkrafturinn sem verkar á boltar, því verður leguspennan
  
  \[
  \sigma = {F_1^{''} \over bd} 
  \Rightarrow 
  F = {4 \sigma_{y} b d r_n \over l} 
  = {
    4 \cdot 210 MPa \cdot 5 mm \cdot 8 mm \cdot 28 mm 
    \over 
    200mm
  } 
  = \underline{4704N}
  \]

  \begin{figure}
    \centering
    \includegraphics[width=0.5\textwidth]{2x8}
    \caption{Annað reiknað tilvik}
    \label{fig::2x8}
  \end{figure}

\end{itemize}

Eftir þessa breytingu fara leguspennur ekki í flot við kraft lægri en $F_{y,p}$ og því eru forsendurnar uppfylltar. 
Aftur á móti munar ekki miklu á kraftinum sem veldur floti í legum og $F_{y,p}$. Því er þriðja hönnunin gerð.

\subsubsection{Þriðja tilvik}

Aðeins flóknari hönnun er gerð í þriðja skiptið. 
Þá eru fjórir boltar notaðir í stað tveggja, \sbr mynd \ref{fig::4x6}. 
Vegna samhverfu dugir að reikna aðeins fyrir aðra hlið stykkisins og aðeins einn bolta þegar kemur að skerspennum og leguspennum.

\begin{itemize}
\item Beygjaspenna á yfirborði við göt
  
  Sá kraftur sem veldur floti fæst með jöfnu \ref{eq::1}. 
  Eina breyting frá fyrri tilvikum er flatartregðuvægið, í þetta skiptið þarf að nota reglu Steiner's. 
  Út frá því fæst að flatartregðuvægið er $I = {1 \over 12}bh^3 - 2\left({1 \over 12}bd^3)\right)$. 
  Flatartregðuvægið er því
  
  \[
  F = \frac{4 \sigma I}{l h} 
  = 
  {
    4 \cdot 210 MPa \cdot 
    \left(
      {1 \over 12} \cdot 5 mm \cdot 
      \left(
        50mm
      \right)^3 
      - 2 \cdot 
      \left(
        {1 \over 12} \cdot 5 mm \cdot 
        \left(
          6mm
        \right)^3
      \right)
    \right) 
    \over 
    (160 mm + 12 mm) \cdot 50 mm
  } 
  = \underline{5070N}
  \] 
  
\item Skerspenna í bolta
  
  Vegna samhverfu bitans fæst sami kraftur í alla bolta og því nóg er að taka einungis fyrir einn bolta.
  
  \[
  F_1^{''} 
  = F_2^{''} 
  = F_3^{''} 
  = F_4^{''} 
  = {Mr_n \over 4r_n^2} 
  = {{F \over 2}l \over 4 r_n} 
  \]
  
  Í þessu tilviki er $r_n = \sqrt{(40mm - 12mm)^2 + (25mm - 13mm)^2} = 30.5 mm$. 
  Því fæst að sá kraftur sem veldur floti í boltum sé 
  \[
  \sigma = {F_1^{''} \over A_b} 
  \Rightarrow F 
  = {8 S_{sy} A_b r_n \over l} 
  = {8 \cdot {660MPa \over \sqrt{3}} \cdot 
    {
      \pi 
      \left(
        6mm
      \right)^2 
      \over 
      4
    } 
    \cdot 30.5 mm \over 200 mm
  } 
  = \underline{13144N}
  \]
  
  
\item Leguspenna í efni
  
  Líkt og áður þá er það skerkrafturinn sem verkar á boltana sem verkar á legu efnisins, því verður leguspenna í efninu
  
  \[
  \sigma = {F_n^{''} \over bd} 
  \Rightarrow F
  = {8 \sigma_{y} b d r_n \over l} 
  = {
    8 \cdot 210 MPa \cdot 5 mm \cdot 6 mm \cdot 30.5 mm 
    \over 
    200mm
  } 
  = \underline{7690N}
  \]

\end{itemize}

Við þessa hönnun fæst að sá kraftur sem veldur floti er töluvert hærri en viðmiðunarkrafturinn $F_{y,p}$ í öllum tilvikum. 
Því er þessi hönnun valin og smíðað eftir teikningu sem sjá má á mynd \ref{fig::smidamynd}.


\begin{figure}
  \centering
  \includegraphics[width=0.5\textwidth]{4x6}
  \caption{Þriðja reiknað tilvik}
  \label{fig::4x6}
\end{figure}

\begin{figure}
	\centering
	\includegraphics[width=1.4\linewidth, angle = 90]{Samsetning}
	\caption{Málsett smíðateikning sem smíðað var eftir}
	\label{fig::smidamynd}
\end{figure}