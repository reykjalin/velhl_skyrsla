\documentclass[12pt, draft, isbabel]{rureport}


\begin{document}

\maketitle

\tableofcontents

\listoffixmes

\clearpage

\section{Inngangur}\label{ch::inngangur}

Markmið verkefnisins er að hanna boltasamsetningu sem halda skal þremur stál stöngum saman. 
Við hönnun skal stærð, fjöldi og staðsetning bolta útfærð svo að samsetningin beri sem mest álag F, álagstilvik er sýnt á mynd \ref{fig::forsendur}. 
Athuga skal að myndin sýnir ekki heppilegt tilvik samsetningar. 

Nota skal 8.8 bolta og reikna með að efni í stöngum sé stál AISI 1020 HR eða sambærilegt. 
Reikna skal öll möguleg tilvik á að samsetningin eða stangirnar gefi sig. Samsetningin verður síðar prófuð og bera skal útkomu prófunarinnar saman við reiknuð gildi á mestu spennu. 
Ef útkoma prófana stangast á við reiknuð gildi skal gera grein fyrir ástæðum þess. 
Reikna má með því að undirstöðurnar eru staddar 1 cm frá enda stanga.

\begin{figure}[b]
	\centering
	\includegraphics[width=1\textwidth]{forsendur}
	\caption{\fxnote{Veit ekki alveg hvað á að standa hér}}
	\label{fig::forsendur}
\end{figure}

\section{Reikningar}\label{ch::reikningar}




\section{Niðurstöður}
\label{sec:nidurstodur}

\begin{figure}[b]
  \centering
  \includegraphics[width=\linewidth]{alag}
  \caption{Álagsferill í þolprófun}
  \label{fig:alag}
\end{figure}

\printbibliography
\end{document}\\
